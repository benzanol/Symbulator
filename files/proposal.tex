% Created 2022-05-16 Mon 11:50
% Intended LaTeX compiler: pdflatex
\documentclass[11pt]{article}
\usepackage[utf8]{inputenc}
\usepackage[T1]{fontenc}
\usepackage{graphicx}
\usepackage{grffile}
\usepackage{longtable}
\usepackage{wrapfig}
\usepackage{rotating}
\usepackage[normalem]{ulem}
\usepackage{amsmath}
\usepackage{textcomp}
\usepackage{amssymb}
\usepackage{capt-of}
\usepackage{hyperref}
\date{\today}
\title{}
\hypersetup{
 pdfauthor={},
 pdftitle={},
 pdfkeywords={},
 pdfsubject={},
 pdfcreator={Emacs 27.2 (Org mode 9.4.4)}, 
 pdflang={English}}
\begin{document}

The purpose of this project is to design an easy to use graphing calculator which will show important properties of a function, such as zeros, intersections, or particular values, similar to desmos. The primary goal of this program is to use exact expressions (\(\sqrt{3}\)) instead of approximate ones (1.732\ldots{}) as desmos does. Furthermore, when using the program, you will be able to select particular properties you want to see, such as the derivative of the function at a particular point, places where a function has a certain y value, etc.

The bulk of the program is written in \href{https://scala-lang.org/}{Scala}, which gets compiled to javascript using \href{https://www.scala-js.org/}{ScalaJS} so that the entire program can run natively in a browser. The library that allows you to type in equations is \href{http://mathquill.com/}{MathQuill}, but the system for graphing them was written from scratch.
The core of the program is a symbolic expression system that can store equations and expressions as objects. On top of this, I built a system that defines a wide array of ways to manipulate expressions in order to do things like simplify them, solve for zeros, solve derivatives, and more. Finally, a parser can interperet the equation typed into each equation field and turn it into an expression object, so that the program can apply these manipulations to the equation, as well as graph it to the screen.

The finished project looks similar to desmos in terms of layout, with fields that allow you to type equations on the left and a graph on the right. The graph allows you to pan and zoom, as well as select individual equations or individual points. For each equation that you type in, it will show you all the information you want to know about the equation, including the zeros, intersections, derivative, integral (if it exists.) There are also some advanced options that allow for more options, such as showing definite integrals and areas between graphs.
\end{document}
